\documentclass{jarticle}

\title{On the Rotating Saddle Trap}
\author{Naoya Ozawa}
\date{\today}

\begin{document}

\maketitle

\section{Abstract}
This document discusses the design of the mechanical "rotating saddle" trap\cite{}, which serves as a demonstration of the Paul trap\cite{PaulNobel}.

\section{Paul Trap}
The Paul trap was invented by W. Paul and it is meant to confine charged particles in a limited region by using an electric field. At first this seems to be simply accomplished by creating an electric field that satisfies
\begin{eqnarray*}
\vec{F} & = & q \vec{E} \\
& = & -C \vec{r} \\
& = & -C \left(
\begin{array}{c}
	x \\
	y \\
	z
\end{array} \right)
\end{eqnarray*}
where $C$ is a constant diagonal, which corresponds to the electric potential of the form
\begin{eqnarray*}
\phi & \propto & \alpha x^2 + \beta y^2 + \gamma z^2 \, (\alpha>0,\ \beta>0,\ \gamma>0) \\
\end{eqnarray*}
since
\begin{eqnarray*}
\vec{E} & = & \nabla \phi \\
& \propto & \left(
\begin{array}{c}
	2\alpha x \\
	2\beta y \\
	2\gamma z
\end{array} \right).
\end{eqnarray*}
However, it turns out that this is not the case, since the Gauss' law requires that
\begin{eqnarray*}
\Delta \phi & = & 2 (\alpha + \beta + \gamma) = 0
\end{eqnarray*}
and this can not be simultaneously satisfied with $\alpha>0, \beta>0, \gamma>0$. Thus it is an obvious fact that a charged particle cannot be trapped by a static electric field alone. \\

When the parameters $\alpha = \beta = 1, \gamma = -2$ are chosen, this does not conflict with the Gauss' law, so the electric field corresponding to
\begin{eqnarray*}
\phi & \propto & x^2 + y^2 - 2z^2
\end{eqnarray*}
could be formed statically. By expressing this as
\begin{eqnarray*}
\phi & = & \frac{\phi_0}{2} \left( \frac{\rho^2}{\rho_0^2} - \frac{z^2}{z_0^2} \right)
\end{eqnarray*}
it can be understood that this kind of potential could be formed by placing one "ring" electrode shaped as $z = \pm \frac{z_0}{\rho_0}\sqrt{ \rho^2 + \rho_0^2}$ and two "endcap" electrodes shaped as $\rho = \frac{\rho_0}{z_0}\sqrt{z^2 + z_0^2}$ and applying the voltage $\frac{\phi}{2}$ to them. Note that $r_0^2 = 2z_0^2$ holds here.


\begin{thebibliography}{9}
	\bibitem{PaulNobel}

\end{thebibliography}


\end{document}
