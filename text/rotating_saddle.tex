\documentclass{article}

\title{On the Rotating Saddle Trap}
\author{Naoya Ozawa}
\date{\today}

\begin{document}

\maketitle

\section{Abstract}
This document discusses the design of the mechanical "rotating saddle" trap\cite{}, which serves as a demonstration of the Paul trap\cite{PaulNobel}.

\section{Paul Trap}

\subsection{The Equation of Motion}
The Paul trap was invented by W. Paul and it is meant to confine charged particles in a limited region by using an electric field. At first this seems to be simply accomplished by creating an electric field that satisfies
\begin{eqnarray*}
\vec{F} & = & q \vec{E} \\
& = & -C \vec{r} \\
& = & -C \left(
\begin{array}{c}
	x \\
	y \\
	z
\end{array} \right)
\end{eqnarray*}
where $C$ is a constant diagonal, which corresponds to the electric potential of the form
\begin{eqnarray*}
\phi & \propto & \alpha x^2 + \beta y^2 + \gamma z^2 \, (\alpha>0,\ \beta>0,\ \gamma>0) \\
\end{eqnarray*}
since
\begin{eqnarray*}
\vec{E} & = & \nabla \phi \\
& \propto & \left(
\begin{array}{c}
	2\alpha x \\
	2\beta y \\
	2\gamma z
\end{array} \right).
\end{eqnarray*}
However, it turns out that this is not the case, since the Gauss' law requires that
\begin{eqnarray*}
\Delta \phi & = & 2 (\alpha + \beta + \gamma) = 0
\end{eqnarray*}
and this can not be simultaneously satisfied with $\alpha>0, \beta>0, \gamma>0$. Thus it is an obvious fact that a charged particle cannot be trapped by a static electric field alone. \\

When the parameters $\alpha = \beta = 1, \gamma = -2$ are chosen, this does not conflict with the Gauss' law, so the electric field corresponding to
\begin{eqnarray*}
\phi & \propto & x^2 + y^2 - 2z^2
\end{eqnarray*}
could be formed statically. By expressing this as
\begin{eqnarray*}
\phi & = & \frac{\phi_0}{2} \left( \frac{\rho^2}{\rho_0^2} - \frac{z^2}{z_0^2} \right)
\end{eqnarray*}
it can be understood that this kind of potential could be formed by placing one "ring" electrode shaped as $z = \pm \frac{z_0}{\rho_0}\sqrt{ \rho^2 + \rho_0^2}$ and two "endcap" electrodes shaped as $\rho = \frac{\rho_0}{z_0}\sqrt{z^2 + z_0^2}$ and applying the voltage $\phi_0$ to the "ring" while grounding the "endcaps". Note that $r_0^2 = 2z_0^2$ holds here. \\

A positive charge cannot be stably trapped inside this potential statically since $\gamma < 0$ and it will drift away in the $z$ direction as soon as there is some perturbation. Instead, by using an AC voltage such that $\phi_0 = \phi_0(t) = V_{dc} + V_{ac} \cos{\Omega t}$, the potential will become a "flapping" 3D saddle where the uphill and downhill direction interchange at a frequency $\Omega$. This kind of potential will thus yield the equation of motion for the ion of mass $m$ and charge $q$ as
\begin{eqnarray*}
m \frac{d^2}{dt^2} \left(
\begin{array}{c} 
	\rho \\
	z
\end{array} \right) & = & -q \nabla \phi \\
& = & -q \frac{ V_{dc} + V_{ac} \cos{\Omega t} }{2} \left(
\begin{array}{c}
	\frac{2\rho}{\rho_0^2} \\
	-\frac{2z}{z_0^2}
\end{array} \right) \\
& = & -\frac{q}{\rho_0^2} \left( V_{dc} + V_{ac} \cos{\Omega t} \right) \left(
\begin{array}{c}
	\rho \\
	-2z
\end{array} \right).
\end{eqnarray*}
By defining dimensionless parameters as
\begin{eqnarray*}
\tau & = & \frac{\Omega t}{2} \\
a_z & = & -\frac{8eV_{dc}}{m\rho_0^2 \Omega^2} \\
a_\rho & = & \frac{4eV_{dc}}{m\rho_0^2 \Omega^2} \\
q_z & = & \frac{4eV_{ac}}{m\rho_0^2 \Omega^2} \\
q_\rho & = & -\frac{2eV_{ac}}{m\rho_0^2 \Omega^2}
\end{eqnarray*}
the equation of motion could be rewritten as


\subsection{The Movement of the Ion}




\section{The Rotating Saddle Trap}

\subsection{The Equation of Motion (for a Point-Like Ball)}
A demonstration of the Paul trap could be done by using a "mechanical" saddle. Since it is technically demanding to create a saddle potential that would flap up and down like the Paul trap potential, we instead make a rotating saddle potential. By considering the 2D version of the saddle electric potential ($\alpha = 1,\, \beta = -1,\, \gamma = 0$) and replacing the electric potential $\phi_0$ with the gravitational potential $mgh_0$, the potential surface could be described as
\begin{eqnarray*}
\phi & = & \frac{mgh_0}{\rho_0^2} \left( {x'}^2 - {y'}^2 \right)
\end{eqnarray*}
which rotates according to 
\begin{eqnarray*}
\left(
\begin{array}{c}
	x' \\
	y' \\
	z'
\end{array} \right) & = & \left(
\begin{array}{c}
	\cos{\Omega t} & \sin{\Omega t} & 0 \\
	-\sin{\Omega t} & \cos{\Omega t} & 0 \\
	0 & 0 & 1
\end{array} \right) \left(
\begin{array}{c}
	x \\
	y \\
	z
\end{array} \right).
\end{eqnarray*}
This would yield the equation
\begin{eqnarray*}
m \frac{d^2}{dt^2} \left(
\begin{array}{c}
	x \\
	y \\
	z
\end{eqnarray*} & = & -\frac{2mgh_0}






\begin{thebibliography}{9}
	\bibitem{PaulNobel}

\end{thebibliography}


\end{document}
